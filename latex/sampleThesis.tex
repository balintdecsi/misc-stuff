% LaTeX mintafájl szakdolgozat és diplomamunkáknak az
% SZTE Informatikai Tanszekcsoportja által megkövetelt
% formai követelményeinek megvalósításához
% Modositva: 2011.04.28 Nemeth L. Zoltan
% A fájl használatához szükséges a magyar.ldf 2005/05/12 v1.5-ös vagy késõbbi verziója
% ez letölthetõ a http://www.math.bme.hu/latex/ weblapról, a magyar nyelvû szedéshez
% Hasznos információk, linkek, LaTeX leirasok a www.latex.lap.hu weboldalon vannak.
%


\documentclass[12pt]{report}

%Magyar nyelvi támogatás (Babel 3.7 vagy késõbbi kell!)
%\def\magyarOptions{defaults=hu-min}
\usepackage[magyar]{babel}

%Az ékezetes betûk használatához:
\usepackage{t1enc}% ékezetes szavak automatikus elválasztásához
\usepackage[latin2]{inputenc}% ékezetes szavak beviteléhez

% A formai kovetelmenyekben megkövetelt Times betûtípus hasznalata:
\usepackage{times}

%Az AMS csomagjai
\usepackage{textgreek}

%A fejléc láblécek kialakításához:
\usepackage{fancyhdr}

%Természetesen további csomagok is használhatók,
%például ábrák beillesztéséhez a graphix és a psfrag,
%ha nincs rájuk szükség természetesen kihagyhatók.

%Tételszerû környezetek definiálhatók, ezek most fejezetenkent egyutt szamozodnak, pl.
\newtheorem{tét}{Theorem}[chapter]
\newtheorem{defi}[tét]{Definition}
\newtheorem{lemma}[tét]{Lemma}
\newtheorem{áll}[tét]{Statement}
\newtheorem{köv}[tét]{Conclusion}

%Ha a megjegyzések és a példak szövegét nem akarjuk dõlten szedni, akkor
%az alábbi parancs után kell õket definiální:
\theoremstyle{definition}
\newtheorem{megj}[tét]{Note}
\newtheorem{pld}[tét]{Example}

%Margók:
\hoffset -1in
\voffset -1in
\oddsidemargin 35mm
\textwidth 150mm
\topmargin 15mm
\headheight 10mm
\headsep 5mm
\textheight 230mm

\renewcommand{\baselinestretch}{1.5}
\begin{document}
	
\selectlanguage{english}

%A FEJEZETEK KEZDÕOLDALAINAK FEJ ES LÁBLÉCE:
%a plain oldalstílust kell átdefiniálni, hogy ott ne legyen fejléc:
\fancypagestyle{plain}{%
%ez mindent töröl:
\fancyhf{}
% a láblécbe jobboldalra kerüljön az oldalszám:
\fancyfoot[R]{\thepage}
%elválasztó vonal sem kell:
\renewcommand{\headrulewidth}{0pt}
}

%A TÖBBI OLDAL FEJ ÉS LÁBLÉCE:
\pagestyle{fancy}
\fancyhf{}
\fancyhead[L]{Single Cell analysis using regression}
\fancyfoot[R]{\thepage}


%A címoldalra se fej- se lábléc nem kell:
\thispagestyle{empty}

\begin{center}
\vspace*{0.4cm}
{\Large\bf University of Szeged}

\vspace{0.33cm}

{\Large\bf Department of Biotechnology}

\vspace*{2.0cm}


{\LARGE\bf Patch clamp automation \\ and its data structuring}


\vspace*{1.8cm}

{\Large Thesis sample}
% vagy {\Large Szakdolgozat}

\vspace*{1.5cm}

{\large \emph{Written by:} \\
	\bf{Bálint, Décsi} \\
}{\large Molecular Bionic Engineering\\ BSc student}

\vspace*{1.6cm}

%Értelemszerûen megváltoztatandó:
{\large
\begin{tabular}{c@{\hspace{3cm}}c}
 \multicolumn{2}{c}{\emph{Supervisors:}}\\
\bf{G\'{a}bor R\'{a}khely, PhD Habil}  &\bf{name}\\
Head of Department    &position\\
Department of Biotechnology &position\\
University of Szeged   &Szeged\\

\end{tabular}
}

\vspace*{1cm}

{\Large
Szeged, 2019
}
\end{center}

%A \chapter* parancs nem ad a fejezetnek sorszámot
\chapter*{Task}
%A tartalomjegyzékben mégis szerepeltetni kell, mint szakasz(section) szerepeljen:
\addcontentsline{toc}{section}{Task}

The aim of my thesis is to give a detailed overview of the evolution of patch clamp automation techniques, focusing on recent developments in the last decade. The state-of-the-art researches in this field are structured on image-based solutions. Furthermore, I describe the plug-ins I developed for a patch clamp automation software (Autopatcher), which are designed to visualize data collected by the software and pipeline them into a database where it can be requested from later.
*Something like this, but with a bit more words, with references etc.*


\chapter*{Abstract}
\addcontentsline{toc}{section}{Abstract}

*Detailing how autopatching developed from 2012 to nowadays, the main milestones (e.g. first image-guided software) and how the functionality of the plug-ins*

%A tartalomjegyzék:
\tableofcontents

\chapter*{List of abbreviations}

*only if needed*

\chapter*{Introduction}
\addcontentsline{toc}{section}{Introduction}

\begin{itemize}
	\item starting from main idea of patch clamping (Neher and Sakmann, 1976), what it is for and is relevance in neurobiology
	\item describing steps of patch clamping (penetrating through dura mater, regional pipette localization, neuron hunting, gigaseal formation, break-in)
	\item weak points of human-controlled patching -> leading to importance of automation and thus scaling
	\item how automation appeared in image-guided processes as well
\end{itemize}

\chapter{Automated blind patching}

*Here the content will be based on the following articles (each one from the articles that Krisztián gave me) Kondaramaiah et al. (2012), Desai et al. (2015) and Li et al. (2017)*

\section{Methods}
\subsection{Basic flowchart of the algorithm}
\subsection{Regulating pressure}
\begin{itemize}
	\item *flowcharts and exact values of p depending on which state the process is in*
	\item *conclusions, i.e. what happens if standard values are changed, how clogging can be avoided via increasing p
\end{itemize}
\subsection{Resistance monitoring}
*how the software is reacting to certain values of R in certain stages of patching, (with most detail in gigaseal formation)
\section{Disadvantages (leading the context towards image-based algorithms)}
*biased towards bigger cells)*

\chapter{Automated image-guided patch clampingn}
*3D rendering of stacks, two-photon vs DIC, insertion of vectors, structure-function correlation*
*Also Autopatcher, how working with it is like (though now it seems that I won’t be able to see that because of covid-19) and leading to describing plug-ins

\chapter{Data visualization and structuring of the autopatching software}
*Yet its content here depends on until where can I get with these tasks*

\end{document}